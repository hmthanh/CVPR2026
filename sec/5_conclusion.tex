\section{Conclusion}
\label{chap:Conclusion}

Although our rule-based method achieves performance comparable to modern deep learning approaches while requiring significantly less training time—only about 17 minutes compared to several hours—this does not diminish the importance of deep learning models. In fact, our analysis across multiple datasets shows that deep learning methods, particularly those employing attention mechanisms, are better suited to knowledge graphs with diverse relation types. For instance, on datasets like FreeBase that exhibit a rich relational structure, models such as KBGAT demonstrate clear advantages by effectively leveraging relational embeddings to capture complex graph structures.

Conversely, rule-based approaches show strong performance on datasets with many similar or inverse triples. However, when such redundant patterns are removed—as in filtered datasets like FB15k-237 or WN18RR—the effectiveness of rule-based methods decreases, since they often depend on recurring structural motifs. Deep learning models, by embedding entities and relations into continuous vector spaces, can better generalize to these more challenging settings.

An advantage of rule-based methods lies in their interpretability and low computational cost during training. Nevertheless, they often require extensive inference time, as predictions involve evaluating all learned rules. Deep learning models, in contrast, leverage trained weights for efficient probabilistic inference, though they generally lack transparency and are resource-intensive during training.

Regarding our two proposed algorithms for integrating new knowledge into the graph, experimental results show that they significantly outperform baseline deep learning methods. This highlights the effectiveness of our approach in dynamic knowledge graph settings.

Moreover, our findings suggest that current graph embedding techniques may benefit from reconsidering the dimensional encoding of entities and relations. Since these components capture different aspects of knowledge, assigning them the same vector space may limit expressiveness. Future work should explore variable-dimension embeddings and optimal dimensionality ratios.

Finally, we emphasize the importance of incorporating temporal information. In real-world applications, the meaning of facts can shift significantly over time. Integrating temporal features into attention mechanisms represents a promising direction to improve semantic accuracy and ensure more robust temporal reasoning in knowledge graph models.


%Trong phần này chúng tôi sẽ trình bày các kết quả đạt được của mô hình chúng tôi, cũng như những phân tích của chúng tôi
%trên các kết quả của các tập dữ liệu khác nhau để giải thích những điểm tốt và điểm cần cải thiện trên mô hình của chúng tôi trên tập dữ liệu đó. Từ đó chúng tôi xác định những hướng nghiên cứu để cải tiến trong tương lai.
%% chúng tôi cố gắng tìm hiểu các đặc trưng của các bộ dữ liệu tương ứng để cố gắng lý  giải thích  tại sao mô hình của chúng tôi hoặc các công trình khác có được kết quả tốt trên tập dữ liệu tương ứng đó.
%% Những kết quả của hai đề xuất của chúng tôi cũng như các dịnh hướng nghiên cứu của chúng tôi trong tương lai.
%
%Mặc dù kết quả chúng tôi cho thấy phương pháp dựa trên luật của chúng tôi có hiệu suất tương đương với các mô hình học sâu hiện đại (state-of-art) và có ưu thế vượt trội trong thời gian đào tạo khoảng 17 phút so với thời gian hàng giờ của phương pháp học sâu khác nhưng không phải là các mô hình học sâu này không đáng nghiên cứu. Chúng tôi cũng nhận thấy, với tập dữ liệu có nhiều loại quan hệ khác nhau như FreeBase, mô hình KBGAT nhờ sử dụng cơ chế chú ý đạt được kết quả tốt hơn so với tập WorldNet với số lượng các loại quan hệ ít hơn.


% Ngược lại đối với các phương pháp dựa trên học sâu lại có ưu thế rất lớn trong các tập dữ liệu này do có thể dễ dàng tính toán độ gần của các luật mới cần đánh giá so với các luật đã học từ đó có một kết quả khá tốt.
% Do đó chúng tôi cũng sẽ tiếp tục nghiên cứu các phương pháp học sâu và sẽ dùng phương pháp này làm đường cơ sở (base line) để so sánh với các nghiên cứu của chúng tôi trong tương lai.
% Một điểm yếu nữa của mô hình đựa trên luật của chúng tôi là mặc dù thời gian học là vượt trội nhưng thời gian để tính toán đưa ra đự đoán khá lâu do phải duyệt qua tất cả các luật được sinh ra mới có thể đưa ra dự đoán.
% Không giống như các phương pháp nhúng đồ thị khác thao tác này có thể dễ dàng tính toán.
